\documentclass[letterpaper]{article}
\usepackage{aaai}
\usepackage{times}
\usepackage{helvet}
\usepackage{courier}
% %%%%%%%%%%%%%%%%%%%%%%%%%%%%%%%%%%%%%%%%%%%%%%%%%%%%%%
% PDFMARK for TeX and GhostScript
% Uncomment and complete the following for metadata if
% your paper is typeset using TeX and GhostScript (e.g
% if you use .ps or .eps files in your paper):
% \special{! /pdfmark where
% {pop} {userdict /pdfmark /cleartomark load put} ifelse
% [ /Author (John Doe, Jane Doe)
% /Title (Paper Title)
% /Keywords (AAAI, artificial intelligence)
% /DOCINFO pdfmark}
% %%%%%%%%%%%%%%%%%%%%%%%%%%%%%%%%%%%%%%%%%%%%%%%%%%%%%%
% PDFINFO for PDFTeX
% Uncomment and complete the following for metadata if
% your paper is typeset using PDFTeX
\pdfinfo{
  /Title (Solving an Exam Scheduling Problem Using a Genetic Algorithm)
  /Subject (Input The Proceedings Title Here)
  /Author (Dave, Kordalewski;
           Caigu, Liu;
           Kevin, Salvesen;)
} 
% %%%%%%%%%%%%%%%%%%%%%%%%%%%%%%%%%%%%%%%%%%%%%%%%%%%%%%
% Uncomment only if you need to use section numbers
% and change the 0 to a 1 or 2
% \setcounter{secnumdepth}{0}
% %%%%%%%%%%%%%%%%%%%%%%%%%%%%%%%%%%%%%%%%%%%%%%%%%%%%%%

\title{Solving an Exam Scheduling Problem Using a Genetic Algorithm}
\author{Dave Kordalewski \\ davekordalewski@gmail.com
   \And Caigu Liu \\liucaigu@gmail.com
   \And Kevin Salvesen \\ kevin.salvesen@gmail.com}
   
\nocopyright

\begin{document}
\maketitle

\begin{abstract}
  This concise, one-paragraph summary should describe the general thesis
  and conclusion of your paper. A reader should be able to learn the purpose
  of the paper and the reason for its importance from the abstract.
\end{abstract}

\section{The Problem}
  The problem we investigate is a sort of scheduling problem. 
  We are attempting to find the most satisfactory choice of when
  and where to hold exams, given a background of student course loads. 

  An instance of this scheduling problem consists of a number of days
  on which exams can be scheduled ($d$), the number of time slots in which
  an exam can be scheduled on any day ($t$), a set of rooms ($R$), a set of
  courses ($C$), and a set of students ($S$), each of whom ($s$) has a
  particular course load, some subset of the courses ($L_s$).

  A schedule, then, is a mapping from courses to rooms at times, 
  which we can express like this:
  
  \[ C \rightarrow \left\{(r,\ a,\ b)\ |\ r\in R,\ a\in \{1..d\},\ b\in \{1..t\}\right\} \]
  
  The total number of different possible schedules in any problem instance is
  $|C|^{|R|dt}$ which is typically far too large for brute force search.
  For example, one instance that we examine (scheduling the fall semester
  exams at the University of Toronto in 2009) involves 603 courses, 7 days,
  8 times, 43 rooms, and 21945 students. This instance admits approximately 
  $10^{6695}$ different possible schedules.

  Typically, with relatively loose constraints on the number of rooms and 
  particular schedules of students, it is not difficult to find some consistent
  schedule; that is, one where no student is asked to write 2 exams
  simultaneously and no room has 2 exams occur in it simultaneously.

  Rather than worry about hard constraints, we prefer a framework of soft 
  constraints, where we try to find a schedule that makes the students and 
  invigilators happiest overall, allowing the possibility that some student 
  or room is left with an impossible exam schedule, which can, in the real 
  world, be dealt with on an individual basis.
  
  The timetable that a student $s \in S$ has under some particular schedule 
  $K$ may be represented in this way:
  
  \[ TT_s(K) = \left\{(a, b)\ |\ \exists r \in R, \exists c \in L_s, K(c)=(r, a, b) \right\} \]
  
  which may, in general, be a multiset.
  
  Similarly, the timetable for a room $r \in R$ is
  
  \[ TT_r(K)=\left\{(a, b)\ |\ \exists c\in C,\ K(c)=(r, a, b)\right\} \]
  
  We define two quality functions, mapping schedules to real numbers in [0,1],
  one for students and one for rooms. These are meant to capture how much they
  "like" the schedule under consideration. (For instance, a student will rate
  poorly any schedule where she has 2 consecutive exams, and very poorly any 
  schedule which expects her to take two exams at the same time.)
  
  \[ Q_s(K)=f\left(TT_s(K)\right):\textit{ Schedules}\rightarrow [0,\ 1] \]
  \[ Q_r(K)=g\left(TT_r(K)\right):\textit{ Schedules}\rightarrow [0,\ 1] \]
  
  We will consider later how to define these functions in a reasonable way.

  The quality of a schedule, then, is given by
  
  \[ Q(K)= \frac{\left(w_s\left(\sum_{s \in S}{\frac{Q_s(K)}{|S|}}\right) +
     w_r\left(\sum_{r\in R}{\frac{Q_r(K)}{|R|}} \right)\right)}{(w_s+w_r)} \]
  
  where $w_s$ and $w_r$ are the weightings we can use to indicate that we 
  care somewhat more about students preferences than rooms, giving a quality
  (or fitness, in the terminology of genetic algorithms) in the range [0,1] for
  any schedule.
  
  Evaluating this function Q can be computationally expensive when $|S|$ and $|R|$
  are large. Our work examines how to find a schedule with relatively high fitness
  considering that we will want to do this with as few evaluations of Q as possible.

  We examine, first and foremost, a genetic algorithm approach to solving this problem,
  but also touch on a few other methods.

\section{Approach and Implementation}
  Tools, algorithms, APIs, etc. that we used.

\section{Evaluation}
  How we determined the success of our project. Did we solve it? Effectively?
  Labelled images, graphs and/or tables can be useful.
  
\section{Conclusion}
  What have you learned because of your project? What can we learn from your
  project? Did it give you ideas as to things you might like to try next?

% \bibliography{Bibliography-File}
% \bibliographystyle{aaai}
\end{document}