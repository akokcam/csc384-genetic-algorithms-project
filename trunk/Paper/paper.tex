\documentclass[letterpaper]{article}
\usepackage{aaai}
\usepackage{times}
\usepackage{helvet}
\usepackage{courier}
% %%%%%%%%%%%%%%%%%%%%%%%%%%%%%%%%%%%%%%%%%%%%%%%%%%%%%%
% PDFMARK for TeX and GhostScript
% Uncomment and complete the following for metadata if
% your paper is typeset using TeX and GhostScript (e.g
% if you use .ps or .eps files in your paper):
% \special{! /pdfmark where
% {pop} {userdict /pdfmark /cleartomark load put} ifelse
% [ /Author (John Doe, Jane Doe)
% /Title (Paper Title)
% /Keywords (AAAI, artificial intelligence)
% /DOCINFO pdfmark}
% %%%%%%%%%%%%%%%%%%%%%%%%%%%%%%%%%%%%%%%%%%%%%%%%%%%%%%
% PDFINFO for PDFTeX
% Uncomment and complete the following for metadata if
% your paper is typeset using PDFTeX
\pdfinfo{
  /Title (Solving an Exam Scheduling Problem Using a Genetic Algorithm)
  /Subject (Input The Proceedings Title Here)
  /Author (Dave, Kordalewski;
           Caigu, Liu;
           Kevin, Salvesen;)
} 
% %%%%%%%%%%%%%%%%%%%%%%%%%%%%%%%%%%%%%%%%%%%%%%%%%%%%%%
% Uncomment only if you need to use section numbers
% and change the 0 to a 1 or 2
% \setcounter{secnumdepth}{0}
% %%%%%%%%%%%%%%%%%%%%%%%%%%%%%%%%%%%%%%%%%%%%%%%%%%%%%%

\title{Solving an Exam Scheduling Problem Using a Genetic Algorithm}
\author{Dave Kordalewski \\ davekordalewski@gmail.com
   \And Caigu Liu \\liucaigu@gmail.com
   \And Kevin Salvesen \\ kevin.salvesen@gmail.com}
   
\nocopyright

\begin{document}
\maketitle

\begin{abstract}
  This concise, one-paragraph summary should describe the general thesis
  and conclusion of your paper. A reader should be able to learn the purpose
  of the paper and the reason for its importance from the abstract.
\end{abstract}

\section{The Problem}
  Consider a university, with all of its students, courses and classrooms. 
  How does one try to build the best possible schedule for an exam session? 
  A good schedule is a schedule where:
  \begin{itemize}
  \item  There are no conflicts, such as two different exams in the same room
         at the same, or a student having two different exams at the same time.
  \item  Students don't have too many exams in a very short time.
  \item  It is bad for a room to have two exams directly one after another,
         as it doesn't give time for teachers and TAs to set up the room.
  \end{itemize}

 Considering a huge university like University of Toronto, with its 70000 students
 and numerous courses, making a good schedule by hand would be very tedious. 

\section{Approach and Implementation}
  Tools, algorithms, APIs, etc. that we used.

\section{Evaluation}
  How we determined the success of our project. Did we solve it? Effectively?
  Labelled images, graphs and/or tables can be useful.
  
\section{Conclusion}
  What have you learned because of your project? What can we learn from your
  project? Did it give you ideas as to things you might like to try next?

% \bibliography{Bibliography-File}
% \bibliographystyle{aaai}
\end{document}