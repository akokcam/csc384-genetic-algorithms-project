\documentclass[letterpaper]{article}
\usepackage{aaai}
\usepackage{times}
\usepackage{helvet}
\usepackage{courier}
\usepackage{url}
% %%%%%%%%%%%%%%%%%%%%%%%%%%%%%%%%%%%%%%%%%%%%%%%%%%%%%%
% PDFMARK for TeX and GhostScript
% Uncomment and complete the following for metadata if
% your paper is typeset using TeX and GhostScript (e.g
% if you use .ps or .eps files in your paper):
% \special{! /pdfmark where
% {pop} {userdict /pdfmark /cleartomark load put} ifelse
% [ /Author (John Doe, Jane Doe)
% /Title (Paper Title)
% /Keywords (AAAI, artificial intelligence)
% /DOCINFO pdfmark}
% %%%%%%%%%%%%%%%%%%%%%%%%%%%%%%%%%%%%%%%%%%%%%%%%%%%%%%
% PDFINFO for PDFTeX
% Uncomment and complete the following for metadata if
% your paper is typeset using PDFTeX
\pdfinfo{
  /Title (Solving an Exam Scheduling Problem Using a Genetic Algorithm)
  /Subject (Input The Proceedings Title Here)
  /Author (Dave, Kordalewski;
           Caigu, Liu;
           Kevin, Salvesen;)
} 
% %%%%%%%%%%%%%%%%%%%%%%%%%%%%%%%%%%%%%%%%%%%%%%%%%%%%%%
% Uncomment only if you need to use section numbers
% and change the 0 to a 1 or 2
% \setcounter{secnumdepth}{0}
% %%%%%%%%%%%%%%%%%%%%%%%%%%%%%%%%%%%%%%%%%%%%%%%%%%%%%%

\title{Solving an Exam Scheduling Problem Using a Genetic Algorithm}
\author{Dave Kordalewski \\ davekordalewski@gmail.com
   \And Caigu Liu \\liucaigu@gmail.com
   \And Kevin Salvesen \\ kevin.salvesen@gmail.com}
   
\nocopyright

\begin{document}
\maketitle

\begin{abstract}
  This concise, one-paragraph summary should describe the general thesis
  and conclusion of your paper. A reader should be able to learn the purpose
  of the paper and the reason for its importance from the abstract.
\end{abstract}

\section{The Problem}
  The problem we investigate is a sort of scheduling problem. 
  We are attempting to find the most satisfactory choice of when
  and where to hold exams, given a background of student course loads. 

  An instance of this scheduling problem consists of a number of days
  on which exams can be scheduled ($d$), the number of time slots in which
  an exam can be scheduled on any day ($t$), a set of rooms ($R$), a set of
  courses ($C$), and a set of students ($S$), each of whom ($s$) has a
  particular course load, some subset of the courses ($L_s$).

  A schedule, then, is a mapping from courses to rooms at times, 
  which we can express like this:
  
  \[ C \rightarrow \left\{(r,\ a,\ b)\ |\ r\in R,\ a\in \{1..d\},\ b\in \{1..t\}\right\} \]
  
  The total number of different possible schedules in any problem instance is
  $|C|^{|R|dt}$ which is typically far too large for brute force search.
  For example, one instance that we examine (scheduling the fall semester
  exams at the University of Toronto in 2009) involves 603 courses, 7 days,
  8 times, 43 rooms, and 21945 students. This instance admits approximately 
  $10^{6695}$ different possible schedules.

  Typically, with relatively loose constraints on the number of rooms and 
  particular schedules of students, it is not difficult to find some consistent
  schedule; that is, one where no student is asked to write 2 exams
  simultaneously and no room has 2 exams occur in it simultaneously.

  Rather than worry about hard constraints, we prefer a framework of soft 
  constraints, where we try to find a schedule that makes the students and 
  invigilators happiest overall, allowing the possibility that some student 
  or room is left with an impossible exam schedule, which can, in the real 
  world, be dealt with on an individual basis.
  
  The timetable that a student $s \in S$ has under some particular schedule 
  $K$ may be represented in this way:
  
  \[ TT_s(K) = \left\{(a, b)\ |\ \exists r \in R, \exists c \in L_s, K(c)=(r, a, b) \right\} \]
  
  which may, in general, be a multiset.
  
  Similarly, the timetable for a room $r \in R$ is
  
  \[ TT_r(K)=\left\{(a, b)\ |\ \exists c\in C,\ K(c)=(r, a, b)\right\} \]
  
  We define two quality functions, mapping schedules to real numbers in [0,1],
  one for students and one for rooms. These are meant to capture how much they
  "like" the schedule under consideration. (For instance, a student will rate
  poorly any schedule where she has 2 consecutive exams, and very poorly any 
  schedule which expects her to take two exams at the same time.)
  
  \[ Q_s(K)=f\left(TT_s(K)\right):\textit{ Schedules}\rightarrow [0,\ 1] \]
  \[ Q_r(K)=g\left(TT_r(K)\right):\textit{ Schedules}\rightarrow [0,\ 1] \]
  
  We will consider later how to define these functions in a reasonable way.

  The quality of a schedule, then, is given by
  
  \[ Q(K)= \frac{\left(w_s\left(\sum_{s \in S}{\frac{Q_s(K)}{|S|}}\right) +
     w_r\left(\sum_{r\in R}{\frac{Q_r(K)}{|R|}} \right)\right)}{(w_s+w_r)} \]
  
  where $w_s$ and $w_r$ are the weightings we can use to indicate that we 
  care somewhat more about students preferences than rooms, giving a quality
  (or fitness, in the terminology of genetic algorithms) in the range [0,1] for
  any schedule.
  
  Evaluating this function Q can be computationally expensive when $|S|$ and $|R|$
  are large. Our work examines how to find a schedule with relatively high fitness
  considering that we will want to do this with as few evaluations of Q as possible.

  We examine, first and foremost, a genetic algorithm approach to solving this problem,
  but also touch on a few other methods.

\section{Approach and Implementation}
  \subsection{Differences From Natural Exam Scheduling Problems}
    The problem we have constructed here is a compromise between the most
    mathematically pure problem with the same sort of properties (which might ignore the
    rooms and related constraints completely) and the problem that must be solved in the
    real world when scheduling thousands of students, which has some important differences
    from this problem.
    
  \subsection{Tools used}
    The program is implemented in Java, to generate possible schedules and to calculate
    their fitness. A Mathematica program is also written to plot graphs, given the data
    generated by the Java program. Additionally, we use Google code 
    subversion\footnote{\url{http://code.google.com/p/csc384-genetic-algorithms-project/}}
    and Google doc to share the work.
    
  \subsection{Fitness Function}
    In order to assign each schedule a fitness, we need to define the functions 
    $q_{student}$ and $q_{room}$. They need to capture the idea of determining how
    much each student or room likes their given timetable. We do this by calculating 
    the penalty incurred by a room, which is a measure of it having properties that 
    are unpleasant to the student or room, and then using the formula
    
    \[ q(TT)=\frac{1}{1+penalty_{TT}} \]
    
    This function will be in the desired range or [0,1].
    
    For students, we accrue a penalty of 50 for every time that is repeated in a 
    timetable, 5 when there are exams in 2 consecutive times, 3 when 2 exams happen 
    on the same day, and 0.5 when the student has exams on consecutive days. This 
    accords reasonably well with real students' hopes for their exam timetables.
    
    Rooms have the same penalty of 50 when it is used by two exams at the same time, 
    5 if it hosts exams in consecutive time slots (reasoning that the invigilators 
    need time to let the students out and in and prepare for the next exam) and a 
    penalty of 0.5 when a room is empty for an entire day between uses (reasoning 
    that the room requires some amount of modifications to be used for exams, and 
    it is difficult to use it for other purposes if it will still be used for exams 
    soon.)
    
  \subsection{Genetic Algorithm Search}
    Although there are many readily available open source genetic algorithm software
    packages available (see JGAP \cite{website:JGAP}, Jenes\cite{website:jenes}) we 
    decided to write a simple, yet general, genetic algorithm package, in Java, that
    can interface with data types designed by others, so long as they implement a few
    necessary methods. The package we wrote is more than adequate for investigating
    the scheduling problem under consideration here.
    
    \subsubsection{General Description}
      [\ldots]
    \subsubsection{Mutation Operator}
      [\ldots]
    \subsubsection{Crossover Operator}
      [\ldots]
  \subsection{Other Search Algorithms}
    \subsubsection{Random Search}
      The random search method simply generates random schedules, evaluates their fitness,
      and remembers the one that it has seen that has highest fitness. This method is meant
      as a baseline against which to compare the other methods. If a search method doesn't
      perform significantly better than random search with the same number of fitness
      evaluations, it is useless.
      
    \subsubsection{Hillclimbing Search}
      The hillclimbing search that we have implemented examines every schedule in the
      neighbourhood of a given starting schedule, and then continues from the one with
      highest fitness, terminating when it has performed as many evaluations as it is allowed
      to, or it has found a schedule whose neighbour are all equal or inferior to it.
      
      The neighbouring schedules are those where only one course meets in either a different
      room, on a different day, or at a different time. This leads to a total of
      $ |C|(d+t+|R|-3) $ schedules in the neighbourhood of any given schedule.
      
      It can be seen that this method is completely deterministic given a starting schedule,
      and will always return a local maximum

    \subsubsection{Mutate Search}
      Mutate Search starts by generating a random schedule. It modifies this schedule (by
      using the same mutate operator made for the genetic algorithm) by 1\% and rolls back if
      the mutation didn't increase the schedule's fitness. This operation is repeated until it has
      performed as many evaluations as it is allowed to.
      This algorithm is a variation of simulated annealing, where instead of accepting a change
      through a probabilistic function, it always accepts a change that leads to a higher fitness.

\section{Evaluation}
  \subsection{Success}
    [\ldots]
    Seeing as our fitness function gives a very high penalty for schedules that have direct
    conflicts (either a student with two exams at the same moment, or two different exams
    in a same room at the same moment), if such a schedule exists given the input data, our
    algorithms will very quickly tend to reject all such schedules.
  \subsection{The Genetic Algorithm's Parameters}
    [\ldots]
  \subsection{Comparison Between the Different Searches}
    [\ldots]
  
\section{Conclusion}
  We come up with a conclusion that in this particular problem, genetic algorithm
  is better than random search, and it is better than hill-climbing if the input 
  is a large data set. But it may not be the best solution for this problem, since
  the MutateSearch performs way better.
  
  %\cite{website:jgap} 
  %\cite{website:jenes} 
  %\cite{website:coderepos} 
\bibliographystyle{aaai}
\bibliography{paperBib}
\end{document}